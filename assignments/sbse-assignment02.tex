\documentclass{pracs}
\usepackage{cases}
\newcommand{\duedate}{Week 7}
\newcommand{\longname}{Assignment 2}
\newcommand{\pracname}{prac2}
\newtheorem{exercise}{Exercise}
\begin{document}
\title{Assignment 2: Optimisation of non-functional properties}
\maketitle
\thispagestyle{fancy}


	Formative, 	Learning objectives (1,2,3),\\
 	Abstraction (4), 	Design (4), 	Communication (4), 	Data (5),	Programming (5)


\begin{center}
Due: \duedate, Weight: 15.0 \% of the course
\end{center}

\sloppy

\section*{Preparation}

This assignment is going to be exciting: actual hands-on optimisation of non-functional properties!

In this course, we have already talked about the optimising code for energy consumption, and we have talked about multi-objective optimisation.

First, it is mandatory that you do some reading: 

\begin{itemize}
\item \texttt{sbse-assignment02-paper.pdf} is a copy of the FSE 2015 paper "Optimizing energy consumption of GUIs in Android apps: a multi-objective approach". You can find the file on the course page. % \url{https://myuni-canvas.adelaide.edu.au/files/3275060/download?download_frd=1}
\item \texttt{sbse-assignment02-slides.pdf} are slides to provide a gentle introduction to the topic. You can find the file on the course page. %, \url{https://myuni-canvas.adelaide.edu.au/files/3275061/download?download_frd=1}
\end{itemize}

There is no need to understand everything, however, we strongly recommend to spend maybe two to three hours as a team with reading both PDFs and looking up terms that you have not heard of before. If something remains really unclear, then please ask via the discussion forums. 

The first assignment has been relatively theoretical. This second assignment is largely hands-on and it will involve programming in Java. Do not be afraid: the necessary Java concepts are super basic. If your Java skills or programming skills in general are a bit rusty, then you might want to spend a couple of hours to remind yourself how to write basic code, e.g., access arrays, use loops, use functions, etc.

Lastly, download the following two files:

\begin{itemize}
\item Nexus 6 experimental data. You can find the file on the course page. %, \url{https://myuni-canvas.adelaide.edu.au/files/3275066/download?download_frd=1}
\item The target application with two graphical user interfaces (inside the Calculator package), and  an optimiser app (called GuiOptimiser) that can serve as a starting point for the optimisation. The optimiser will run the target app in full-screen mode, take a screenshot of its graphical user interface, and save it to disk. This was tested under Windows/Linux/MacOS. You can find the file on the course page.
%Download link: \url{https://myuni-canvas.adelaide.edu.au/files/3275063/download?download_frd=1}
\end{itemize}

In case you cannot take screenshot reliably, try increasing the following value in guioptimiser.GuiOptimiser: \\\texttt{TARGET\_APP\_RUNNINGTIME = 1000; // try 5000 or even longer}.
If you encounter technical problems, please post your problem with a description of what you have tried on the discussion forums. This way, we can help not only you but also others.





\section{Overview}
Assignments should be done in groups consisting of three to four (3--4) students.  If you have problems finding a group partner use the forum to search for group partners or contact the lecturer.

\textbf{For this assignment you will have to hand in a report briefly describing what you have done. In your experiments you will have to write some code -- you will need to include the code in your submission.}

\section{Assignment}

\begin{exercise}
Team work: who has done what? (zero points)
\end{exercise}

We'd like each team member to write one paragraph about what he or she has contributed to this assignment. We will not mark this, and it will not have any effect on the marking of the other exercises. You might now ask "why do this then?" --- well, through this no-stakes approach, we'd like to encourage self-regulation within the group and cooperative learning. You can't lose, you can only win.

%%%%%%%%%%%%%%%%%%%%%%%%%%%%%%%%%%%%%%%%%

\begin{exercise}
Determine the surrogate function (20 points)
\end{exercise}

\textbf{Preparation}

Although it could be quite fun for you to perform the optimisation on actual Android phones, we do not want you to go through some of the troubles that we have had to deal with in the past --- see the lecture on deep parameter optimisation for details.

Instead of using the actual device, you will use a surrogate function that will provide you with reasonably accurate fitness values. To help you with this, Mahmoud has run some experiments. On a Nexus 6, he has recorded the total system's charge consumption\footnote{Simply speaking: Energy = Charge X Voltage, however, we want to keep it simple here.} when the entire screen was set to various colours (for about 30 minutes each time): white, black, red, green, \ldots, while the actual brightness of the screen was set to its maximum. He has done this by sampling (at a rate of about 4 Hz) the amount of charge that is left in the phone.

Remember the comment from the lecture: the Nexus 6 has a screen with organic light-emitting diode (OLED) technology. This means that different colours can consume different amounts of charge.

Have a look inside the spreadsheet with the experimental data. This spreadsheet contains five tables, but only the table named "screenOn" is of interest to us. This table lists all the samples taken over time, and the plot shows the charge consumed of the entire device when various colours were displayed. 
Black can be seen as a base case, that is, it shows the charge consumed by the device when nothing is shown on the screen.\footnote{"nothing is shown" here means that "every pixel is set to black". Still, the screen is on. It is possible to  "deactivate the screen" in Android, but this also causes lots of changes to the scheduling of tasks --- which is something that we (and you) are not interested in here.} 
You can see that red and green use similar amounts of charge, i.e., the draw different currents. Note how showing red and green at the same time make yellow, however, the charge consumption of yellow is not the same as the sum of the energies consumed by both red and green independently.

\noindent\textbf{Part 1}

Determine the average charge consumed for the three colours red, green, and blue. 

Calculate the ``average charge used per hour'' when considering (for each colour) only the first 10, the first 100, and the first 1000 samples, and when considering the entire duration of each respective experiment.\footnote{note that we had conducted one experiment per colour to collect this data} 
You will see that the average consumption varies slightly depending on the duration of the experiment. To do this, use the columns named \texttt{charge\_used\_accumulated}\footnote{This is based on \texttt{CHARGE\_COUNTER\_EXT} with a 8nAh resolution, see \url{https://source.android.com/devices/tech/power/device}.} (the units are nAh for nano Ampere hours) and \texttt{system\_time\_in\_ms}. 

The unit that you use should be mA (milli Ampere), and you should use the scientific notation to report the results.

In total, you need to provide 12 numbers for the three colours and the four sample set sizes.

Notes:
\begin{itemize}
\item Do not forget to deduct the charge consumed for black.
\end{itemize}

Present the calculation in the report, including plots if you decide to use some. This section must not be longer than two pages. In this section, make sure you include the aim of the calculations, the calculations done, and the results and the interpretation of the results.

\noindent\textbf{Part 2}

Write a function \texttt{calculateChargeConsumptionPerPixel} in Java that calculates the charge consumption (== current) of a single pixel. This function should take three integers (in the range 0\ldots255) as parameters: the pixel's red, green, and blue values.

Integrate this function into our provided framework so that you can read image files and calculate the charge consumption --- base your calculations on the per-pixel consumption of the Nexus 6 (based on the maximum number of samples available in the spreadsheet), even though your images might have a different resolution. 

To demonstrate that your code works, calculate the charge consumption for three images (or screenshots) of your choice. 

For this part, also submit these three images, the number of pixels in the image, and the charge consumption that you calculated.

Notes:
\begin{itemize}
\item The Nexus 6 has a screen resolution of 1,440 x 2,560 = 3,686,400 pixels. 
\item In case you got stuck in Part 1, you can use the following consumption rates when the entire screen shows a particular colour: red 120mA, green 140mA, blue 240mA.
\item Since you will be working with images, here are some pointers at background information, including code and libraries: \\
\url{https://www.tutorialspoint.com/java_dip/understand_image_pixels.htm}\\
\url{https://stackoverflow.com/questions/6524196/java-get-pixel-array-from-image}\\
\url{https://stackoverflow.com/questions/22391353/get-color-of-each-pixel-of-an-image-using-bufferedimages}\\
\url{https://www.dyclassroom.com/image-processing-project/how-to-get-and-set-pixel-value-in-java}
\end{itemize}

In the report, include a copy of your implementation of \texttt{calculateChargeConsumptionPerPixel} and include a brief description in English of how it works.




\begin{exercise}
Single-objective minimisation of charge consumption (40 points)
\end{exercise}

\noindent\textbf{Part 1 - Setup}

Mahmoud's provided framework does not keep track of the best colour settings. Extend his code to implement a proper random search that outputs the best colour settings found. Keep in mind that we want to minimise charge consumption, not maximise it. 

In addition, design and implement two additional and different optimisation algorithms. This can be a hill-climber, a simulated annealing, and a genetic algorithm.

To make things a little bit more realistic, let us assume a customer brings along the following requirement that needs to be satisfied by all colour schemes:

\begin{quote}
The Euclidean distance between the colours (in the three-dimensional colour-space) for the text in the text field and the text field itself should be at least 128.
\end{quote}

What does this do? Well, it is an attempt to ensure that the text in the text field is still readable. In particular, jTextField1 is the background color of jTextField1, and jTextField1TextColor is the foreground (text) color of jTextField1

Only colour configurations that do not violate this constraint should be evaluated via a screenshot. Do not waste resources on evaluating unacceptable solutions.

\noindent\textbf{Part 2 - Experiments}

Now, optimise both target user interfaces with all three algorithms 10 times. The computational budget is 1000 screenshots for each run. 

For each target user interface for for each algorithm report the following:
\begin{itemize}
\item First, for each of the 10 independent runs, record the charge consumption of the best configuration after 10 screenshots (and after 100 screenshots, and after 1000 screenshots). Second, average these numbers for the 10 independent runs.
\item Submit the best final configuration found and the corresponding screenshot.
\end{itemize}

It is voluntary to use fancy visualisations (e.g. box plots or violin plots) to show the performance --- revisit the lecture on working with algorithms for a few additional ideas.

Notes:
\begin{itemize}
\item To represent a solution, it might be natural to consider integer arrays \texttt{int}$[]$.
\item As the experiments can generate a lot of data, you might want to include a function that deletes the screenshot as soon as you have calculated its charge consumption.
\item In case you find it difficult to process the screenshots produced in the PNG format, then you can change the format to BMP in guioptimiser.Capture.
\end{itemize}

In the report, you must briefly describe the algorithms and the variation operators (``mutations'') that you have implemented. For the variation operators, describe why you designed them the way you did. 
You should also briefly present your results. This section should not be longer than three pages.


\begin{exercise}
Multi-objective optimisation (40 points)
\end{exercise}

\noindent\textbf{Part 1 - MOEA Framework}

MOEA Framework is an object-oriented Java-based framework for multi-objective optimisation with metaheuristics, \url{http://moeaframework.org/}. We will use it so that you do not have to design your multi-objective optimiser from scratch.

Download it and run NSGA-II and MOEA/D on the four different problems ZDT1, ZDT2, DTLZ3, and DTLZ4. ZDT1 and ZDT2 are problems with 2 objectives. The DTLZ problems can be varied in the number of objectives---for this assignment, you need to set the number of objectives to 3.

Run each algorithm on each problem 10 times, with a population sizes of 20 and 100, and with a computation budget of 10,000 evaluations $\rightarrow$ 16 combinations that are to be run 10 times.

Report for each combination:
\begin{itemize}
\item The final covered hypervolume: average and standard deviation of the 10 repetitions.
\item Show the final populations as scatter plots in the objective space.\footnote{This means: the ZDT plots will be two-dimensional, and the DTLZ plots will be three-dimensional. The number of generations IS NOT one of these dimensions.}
\end{itemize}

Ideally, do not submit to us 160 plots, but put them together in a way that you think supports the comparison of the different combinations. Again, you might want to revisit the lecture on working with algorithms for a few additional ideas.

Notes:
\begin{itemize}
\item MOEA Framework API: \url{http://moeaframework.org/javadoc/index.html}
\item MOEA Framework user manual (Chapters 2, 3, and 5 are the most important ones): \url{https://sourceforge.net/p/moeaframework/bugs/_discuss/thread/12d4a637/4a31/attachment/MOEAFramework-2.1-ManualFixed.pdf}
\item MOEA Framework examples: \url{http://moeaframework.org/examples.html}
\end{itemize}

In your report, explain in English the changes to the existing code base that you had to apply, and explain what new code you had to write. This should be no longer than one page.

\textbf{General comment.} The purpose of contrasting two quite different population sizes is to show that parameter choices can matter depending on the algorithm and problem. For a while, the topic of "automated algorithm configuration" (AAC) has been quite hot, where algorithms tune other algorithms, and with the relevant papers being cited hundreds of times. What is AAC about? In short, it is about tuning an algorithm and all its parameters and switches such that the resulting configuration works well on a set of (typically unseen) problem instances.
Slightly newer is the idea of configuring an algorithm on-the-fly either at the beginning ("per-instance-configuration") based on information that is easily available, or during a run (going into the direction of "hyperheuristics") based on observed features during the run.
In future Evolutionary Computation Courses, we might dig into this a lot more --- and there is a super tiny chance of AAC appearing in Assignment 3 here --- but for now this is beyond the scope of this Assignment 2.

\noindent\textbf{Part 2 - Multi-objective optimisation of charge consumption}

Imagine a customer who comes to you with an existing system. She asks you to minimise a particular property (here: charge consumption due to the colour profile), but who is also interested in seeing the trade-offs between deviating from the current profile and the charge saved.

In particular, the deviation is measured as: the sum of of Euclidean colour differences for each pixel. This deviation is of course always $\geq0$.

Step 1) Since we do not have a particular client, you need to define your own reference colour scheme. You can define something that you think is visually pleasant, or just take a random one. The only limitation is that not all colours are the same. Do this for both user interfaces.

Step 2) Implement in MOEA Framework the charge consumption and the deviation from your target at a two-objective problem. Both objectives are to be minimised.

Pick one multi-objective algorithm (for example, NSGA-II or SPEA2). Run it once with a total budget of 10,000 screenshots, and with a population size $\mu=20$.

Report on the outcomes (separate for each target user interface):
\begin{itemize}
\item Show as a scatter plot (in 2D) the distribution of the first generation and of the last generation.\footnote{Again, the number of generations IS NOT one of these dimensions.}
\item Provide all 20 screenshots (or fewer, see note below!) of the final population. Also add them to a scatter plot so that we can easily see the trade-offs.
\end{itemize}

Of course, also submit your reference screenshots.

In your report, you should also briefly present your results in no more than three pages.

An important note regarding those 20 screenshots: it can happen that the algorithm returns a set of configurations with fewer than 20 solutions. How come? Well, many multi-objective optimisation algorithms (or frameworks that provide them) decide that they only return as the final solution set the set of non-dominated solutions. This means that if a solution is dominated (for example, because it is worse in all objectives than another solution in the solution set), then this will be omitted in the end. For you, this means that your final solution sets can have anywhere between 1 and 20 non-dominated solutions, so you might be sending us fewer than 20 screenshots.


\section{General procedure for handing in the assignment}
Work should be handed in using the course website. The submission must include:
\begin{itemize}
\item pdf file of your solutions for theoretical assignments
\item all source files (if you created any): if your code does not compile or if it is not sufficiently well documented, \textbf{we will cap the code-related marks at 50\%}
\item descriptions as required in the statement of the exercises
\item a file name README.txt that contains instructions to run the code (if any), the names, student numbers, and email addresses of the group members
\item for each group, there should be only 1 submission
\end{itemize}

\textbf{\emph{Failure to meet all requirements of the 'General procedure for handing in the assignment' will lead to a reduction by twenty (20) marks.}}

Note: there will be a progress presentation session for this assignment. This is a big opportunity for you to get feedback on your progress, and to make last adjustments for the final submission. 
In these progress presentations, we are expecting each group to briefly outline their achievements. You do not have to repeat what the assignment is about -- we all should know this by then. Please do not show us the results of Exercise 2 -- this would spoil the fun! But for the other exercises, you can outline your approach, your results (screenshots), and tell us about your challenges -- maybe we can help you right away.

\end{document}
